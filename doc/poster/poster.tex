\documentclass[final, 8pt]{beamer}

\usepackage[size=custom,width=76.2,height=101.6,scale=0.85]{beamerposter}
\usetheme{confposter}

\newlength{\sepwid}
\newlength{\onecolwid}
\newlength{\twocolwid}
\newlength{\threecolwid}
\setlength{\paperwidth}{30in} % A0 width: 46.8in
\setlength{\paperheight}{40in} % A0 height: 33.1in
\setlength{\onecolwid}{0.3\paperwidth}
\setlength{\twocolwid}{0.464\paperwidth}
\setlength{\threecolwid}{0.708\paperwidth}
\setbeamersize{text margin left=0.05cm,text margin right=0.05cm}

\usepackage{graphicx}
\usepackage{booktabs}
\usepackage{natbib}
% ------------------------------------------------------------------------
% Packages
% ------------------------------------------------------------------------
\usepackage{amsmath}

% ------------------------------------------------------------------------
% Macros
% ------------------------------------------------------------------------
%~~~~~~~~~~~~~~~
% List shorthand
%~~~~~~~~~~~~~~~
\newcommand{\BIT}{\begin{itemize}}
\newcommand{\EIT}{\end{itemize}}
\newcommand{\BNUM}{\begin{enumerate}}
\newcommand{\ENUM}{\end{enumerate}}
%~~~~~~~~~~~~~~~
% Text with quads around it
%~~~~~~~~~~~~~~~
\newcommand{\qtext}[1]{\quad\text{#1}\quad}
%~~~~~~~~~~~~~~~
% Shorthand for math formatting
%~~~~~~~~~~~~~~~
\newcommand\mbb[1]{\mathbb{#1}}
\newcommand\mbf[1]{\mathbf{#1}}
\def\mc#1{\mathcal{#1}}
\def\mrm#1{\mathrm{#1}}
%~~~~~~~~~~~~~~~
% Common sets
%~~~~~~~~~~~~~~~
\def\reals{\mathbb{R}} % Real number symbol
\def\integers{\mathbb{Z}} % Integer symbol
\def\rationals{\mathbb{Q}} % Rational numbers
\def\naturals{\mathbb{N}} % Natural numbers
\def\complex{\mathbb{C}} % Complex numbers
\def\simplex{\mathcal{S}} % Simplex
%~~~~~~~~~~~~~~~
% Common functions
%~~~~~~~~~~~~~~~
\renewcommand{\exp}[1]{\operatorname{exp}\left(#1\right)} % Exponential
\def\indic#1{\mbb{I}\left({#1}\right)} % Indicator function
\providecommand{\argmax}{\mathop\mathrm{arg max}} % Defining math symbols
\providecommand{\argmin}{\mathop\mathrm{arg min}}
\providecommand{\arccos}{\mathop\mathrm{arccos}}
\providecommand{\asinh}{\mathop\mathrm{asinh}}
\providecommand{\dom}{\mathop\mathrm{dom}} % Domain
\providecommand{\range}{\mathop\mathrm{range}} % Range
\providecommand{\diag}{\mathop\mathrm{diag}}
\providecommand{\tr}{\mathop\mathrm{tr}}
\providecommand{\abs}{\mathop\mathrm{abs}}
\providecommand{\card}{\mathop\mathrm{card}}
\providecommand{\sign}{\mathop\mathrm{sign}}
\def\rank#1{\mathrm{rank}({#1})}
\def\supp#1{\mathrm{supp}({#1})}
%~~~~~~~~~~~~~~~
% Common probability symbols
%~~~~~~~~~~~~~~~
\def\E{\mathbb{E}} % Expectation symbol
\def\Earg#1{\E\left[{#1}\right]}
\def\Esubarg#1#2{\E_{#1}\left[{#2}\right]}
\def\P{\mathbb{P}} % Probability symbol
\def\Parg#1{\P\left({#1}\right)}
\def\Psubarg#1#2{\P_{#1}\left[{#2}\right]}
\def\Cov{\mrm{Cov}} % Covariance symbol
\def\Covarg#1{\Cov\left[{#1}\right]}
\def\Covsubarg#1#2{\Cov_{#1}\left[{#2}\right]}
\def\Var{\mrm{Var}}
\def\Vararg#1{\Var\left(#1\right)}
\def\Varsubarg#1#2{\Var_{#1}\left(#2\right)}
\newcommand{\family}{\mathcal{P}} % probability family
\newcommand{\eps}{\epsilon}
\def\absarg#1{\left|#1\right|}
\def\msarg#1{\left(#1\right)^{2}}
\def\logarg#1{\log\left(#1\right)}
%~~~~~~~~~~~~~~~
% Distributions
%~~~~~~~~~~~~~~~
\def\Gsn{\mathcal{N}}
\def\Ber{\textnormal{Ber}}
\def\Bin{\textnormal{Bin}}
\def\Unif{\textnormal{Unif}}
\def\Mult{\textnormal{Mult}}
\def\Cat{\textnormal{Cat}}
\def\Gam{\textnormal{Gam}}
\def\InvGam{\textnormal{InvGam}}
\def\NegMult{\textnormal{NegMult}}
\def\Dir{\textnormal{Dir}}
\def\Lap{\textnormal{Laplace}}
\def\Bet{\textnormal{Beta}}
\def\Poi{\textnormal{Poi}}
\def\HypGeo{\textnormal{HypGeo}}
\def\GEM{\textnormal{GEM}}
\def\BP{\textnormal{BP}}
\def\DP{\textnormal{DP}}
\def\BeP{\textnormal{BeP}}
%~~~~~~~~~~~~~~~
% Theorem-like environments
%~~~~~~~~~~~~~~~

%-----------------------
% Probability sets
%-----------------------
\newcommand{\X}{\mathcal{X}}
\newcommand{\Y}{\mathcal{Y}}
\newcommand{\D}{\mathcal{D}}
\newcommand{\Scal}{\mathcal{S}}
%-----------------------
% vector notation
%-----------------------
\newcommand{\bx}{\mathbf{x}}
\newcommand{\by}{\mathbf{y}}
\newcommand{\bt}{\mathbf{t}}
\newcommand{\xbar}{\overline{x}}
\newcommand{\Xbar}{\overline{X}}
\newcommand{\tolaw}{\xrightarrow{\mathcal{L}}}
\newcommand{\toprob}{\xrightarrow{\mathbb{P}}}
\newcommand{\laweq}{\overset{\mathcal{L}}{=}}
\newcommand{\F}{\mathcal{F}}


\title{Inference of Dynamic Regimes in the Microbiome}
\author{Kris Sankaran and Susan P. Holmes}
\institute{Department of Statistics, Stanford University}

\begin{document}

\begin{frame}

\begin{columns}
\begin{column}{\onecolwid}

\begin{block}{Abstract}
Many studies attempt to characterize microbiome stability and dynamics across
environments \citep{costello2012application, stein2013ecological,
  faust2015metagenomics}. These problems often have a flavor not just of time
series modeling but also of regime detection, the partitioning of time into
intervals with distinct behaviors, a problem with a rich history and application
to speech recognition \citep{fox2011sticky}, finance \citep{lee2009optimal}, EEG
analysis \citep{camilleri2014automatic} and geophysics
\citep{weatherley2002relationship}. In spite of the parallels, regime detection
methods are rarely used in microbiome data analysis.

We distill the core ideas of different regime detection methods, provide example
applications, and share reproducible code, making these techniques more
accessible to microbiome researchers. Ultimately, our goal is to describe types
of regime switching structure that, through careful modeling, can be
incorporated into studies of microbiome dynamics. Code for all examples here is
available at \url{https://github.com/krisrs1128/tsc\_microbiome}.
\end{block}

\begin{block}{Problem description}
Statistical analysis provides succinct representations of complex data. We can
think of the reduced representations as a type of data compression for human
interpretation, and as in any (lossy) compression, there is a choice of what
structure to preserve. Different reductions facilitate different comparisons --
for example, clustering bacteria allows easy comparison of taxonomic categories,
while clustering samples allows a comparison of full community states.

In the regime detection problem, the comparisons we would like to facilitate are
\begin{itemize}
\item For each species, can we assign time intervals to different dynamic
  regimes?
\item Can we define subsets of species which have similar patterns of behavior,
  in terms of these regimes?
\end{itemize}

Conceretely, we may expect that over the time course of a study, individual
species may switch between stable / unstable, increasing / decreasing, or
present / absent regimes, either due to natural ecological dynamics or
experimentally induced perturbations. We would like to detect these alternative
regimes automatically.
\end{block}

\begin{block}{Case study background}
  
\end{block}

\begin{block}{Algorithmic approaches}
\begin{itemize}
\item Classification and Regression Trees [CART]: 
\begin{figure}[ht]
  \centering
  \includegraphics[width=\textwidth]{../figure/rpart_binary}
  \caption{\label{fig:rpart_binary} }
\end{figure}


\item Hierarchical Clustering: 
\end{itemize}

\end{block}

\end{column}

\begin{column}{\onecolwid}

\begin{block}{Parametric modeling}
\begin{itemize}
\item (Sticky) Hidden Markov Models (HMMs):

\item Switching Linear Dynamical Systems (SLDS):

\item Bayesian Analysis of Simultaneous Changepoints (BASIC):
\begin{figure}[ht]
  \centering
  \includegraphics[width=\textwidth]{../figure/basic_bern_heatmap}
  \caption{\label{fig:basic_bern_heamtap} }
\end{figure}

\item Dynamic Tobit Models:
\begin{figure}[!p]
  \centering
  \includegraphics[width=0.9\textwidth]{../figure/abt_scan}
  \caption{\label{fig:abt_scan}}
\end{figure}
\end{itemize}

\end{block}

\begin{block}{Nonparametric modeling}
\begin{itemize}
\item Mixture of Gaussian Processes (mGPs)
\item Sticky Hierarchical Dirichlet Process HMMs (HDP-HMMs)
\end{itemize}
\end{block}

\begin{block}{Case Study}
\end{block}

\end{column}

\begin{column}{\onecolwid}
 \bibliographystyle{plainnat}
 \bibliography{../bibliography.bib}
\end{column}

\end{columns}

\end{frame}
\end{document}
