\documentclass{article}
\usepackage{natbib}
\usepackage{graphicx}

\title{Hierarchical Clustering + HMM Experiments}
\author{Kris Sankaran}

\begin{document}
\maketitle

These are notes on basic approaches to clustering bacteria. The
three views we consider here are,

\begin{itemize}
\item Hierarchically clustering counts of bacteria, using a few different
  distances, with branches sorted according to abundance. We can then interpret
  the associated heatmaps.
\item Computing the PACF for each bacterial time series. This motivates our
  clustering of the innovations (first-order differences) instead of the raw counts.
\item In addition to clustering bacteria, we can try to cluster the time points.
  For this, we run an HMM on each time series.
\end{itemize}


Our motivation comes from two places,
\begin{itemize}
  \item Applying Latent Dirichlet Allocation on samples seems like a useful
    exercise. Can we carry that way of thinking\footnote{At some point these
      notes should include LDA applied to all the bacteria.} across to the
    bacteria (the transposed matrix)?
  \item A footnote in \citep{ren2015achieving}, presents a probabilistic time
    series clustering model as the unified alternative to first clustering home
    sales across census tracts and then running a kalman filter within each
    cluster. The hope is that some simple exploratory analysis in our setting
    might motivate similar types of unified models.
\end{itemize}

There's really nothing too surprising in what follows. We use the
($\text{asinh}$ transformed) antibiotics data set as our case study, filtered to
the $\sim 600$ most variable species. We use this data because we are so
familiar with it, though it should be straightforwards to rerun everything on
the cleanout phyloseq object. There are a few points we've learned in this
analysis which might be worth keeping in mind / thinking through in follow-up
smwork,

\begin{itemize}
\item Choosing an appropriate distance in hierarchical clustering is
  complicated. We get quite different when using (mixtures of) Euclidean and
  Jaccard distances, which offer somewhat different interpretations. Finding a
  way of simplifying the choice or interpretation would be useful.
\item While taxonomic classification is related to the clustering results,
  limiting attention to just taxonomy would miss out on a lot of interesting
  covariation.
\item The cluster sizes resulting from cutting the hierarchical clustering tree
  are highly imbalanced, which makes interpretation somewhat difficult.
\item Looking at the means in each cluster is misleading, due to the hurdle
  (zero point mass + continuous) nature of the counts.
\item Visually navigating hierarchical clustering results is cumbersome, and it
  might be interesting to develop some interactive alternatives.
\item The PACFs of the bacterial time series suggests that the processes are
  more or less markov, and that clustering innovations might be more informative
  than clustering raw counts.
\item Applying HMMs to the counts is trickier than it might seem at first,
  because (1) label-switching makes it difficult to compare results across
  bacteria without some sort of alignment procedure and (2) running HMM on the
  full collection is not appropriate, because different bacteria have very
  different scales, so different bacteria get marked as coming from different
  regimes even if they have similar shapes.
\end{itemize}

\section{Hierarchical clustering}

We can begin to relate bacteria to one another using hierarchical clustering.
There are a few decisions that must be made before we begin interpretation,

\begin{itemize}
  \item What distance should be used to compare bacterial time series?
  \item What agglomoration strategy (complete, average, single link, ...) should we use?
  \item What height should we cut the tree at, in order to inspect cluster centroids?
\end{itemize}

We won't invest too much effort into any of these decisions at this point,
though it's good to keep in mind that they can make a difference. E.g., we could
use a dynamic time warping distance...

\subsection{Euclidean Distance}

Figure \ref{fig:heatmap-euclidean} gives a heatmap of the (species filtered)
antibiotic counts obtained using a hierarchical clustering with default
(complete) link and a euclidean distance between pairs of series. Specifically,
the rows correspond to samples, with three subjects separated into three
horizontal blocks. Within each block, samples are sorted from top to bottom
according to time.

Each column is one species. Columns are sorted according to the leaves of the
hierarchical clustering tree. We have reordered branches so that more abundant
collections of bacteria are placed on the left at any split point.

An indicator of the taxonomy for each column is given by the colors in the bar
along the bottom. Notice that it is not entirely homogeneous -- there are
bacteria with very similar time series but belogning to different parts of the
tree. Further, the higher-order taxonomic labels clump together bacteria with
very different time series shapes. That said, there are many cases where, at the
borders of taxonomic labels, there seems to be some changes in the heatmap,
though there is no cluster boundary drawn.

The vertical lines within the heatmap correspond to the partition defined by
splitting the tree at a prescpecified height, chosen arbitrarily (but with the
general goal of obtaining $\sim 20$ clusters).

\begin{figure}[ht]
  \centering
  \includegraphics[scale=0.15]{figure/heatmap-euclidean}
  \caption{See the main text for a description of how this figure was
    constructed.\label{fig:heatmap-euclidean} }
\end{figure}

In Figures \ref{fig:centroid-euclidean-conditional} and
\ref{fig:centroid-euclidean-presence}, we display the centroids associated with
each cluster. There are two figures because we wanted to avoid just plotting the
means, since the means are thrown off by zero-inflation. Instead, Figure
\ref{fig:centroid-euclidean-conditional} gives the means across all species in
that cluster conditional on their being present in that sample. The raw RSV
values are shown semitransparently in the background. Figure
\ref{fig:centroid-euclidean-presence} shows instead the proportion of species
that have nonzero counts within each cluster. This is the same hurdle-type
decomposition we have studied in our prediction work from before.

\begin{figure}[ht]
  \centering
  \includegraphics[scale=0.15]{figure/centroid-euclidean-conditional}
  \caption{\label{fig:centroid-euclidean-conditional} }
\end{figure}

\begin{figure}[ht]
  \centering
  \includegraphics[scale=0.15]{figure/centroid-euclidean-presence}
  \caption{\label{fig:centroid-euclidean-presence} }
\end{figure}

\subsection{Jaccard Distance}

Below, we regenerate the figures from the previous section, but for the Jaccard
distance. The heatmap looks quite different, since similarities in shapes don't
matter as much as similarities in presence / absence pattern.

\begin{figure}[ht]
  \centering
  \includegraphics[scale=0.15]{figure/heatmap-jaccard}
  \caption{\label{fig:heatmap-jaccard} }
\end{figure}

\begin{figure}[ht]
  \centering
  \includegraphics[scale=0.15]{figure/centroid-jaccard-conditional}
  \caption{\label{fig:centroid-jaccard-conditional} }
\end{figure}

\begin{figure}[ht]
  \centering
  \includegraphics[scale=0.15]{figure/centroid-jaccard-presence}
  \caption{\label{fig:centroid-jaccard-conditional} }
\end{figure}

\subsection{Mixture Distance}

\begin{figure}[ht]
  \centering
  \includegraphics[scale=0.15]{figure/heatmap-mix}
  \caption{\label{fig:heatmap-mix} }
\end{figure}

\begin{figure}[ht]
  \centering
  \includegraphics[scale=0.15]{figure/centroid-mix-conditional}
  \caption{\label{fig:centroid-mix-conditional} }
\end{figure}

\begin{figure}[ht]
  \centering
  \includegraphics[scale=0.15]{figure/centroid-mix-presence}
  \caption{\label{fig:centroid-mix-conditional} }
\end{figure}

\section{Time-series Analysis ideas}

It's natural to consider time series analysis ideas in the context of the
antibiotics study. We'll find that it might make more sense to do analysis on
the differenced (rather than original count) time series.

Figure \ref{fig:pacf} shows the partial autocorrelation function across all bacteria, separated into taxonomic groups.

Figure \ref{fig:pacf_pairs} digs deeper into the order-1 associations, by
plotting neighboring timepoints against each other (for a subset of times).

Figures \ref{fig:heatmap-innovations} and \ref{fig:heatmap-innovations-bin} give
heatmaps obtained by clustering the differenced and
difference-in-presence-absence series, respectively. Figure
\ref{fig:centroid-innovations-conditional} gives the centroids of the
differenced series in this first scenario.

\begin{figure}[ht]
  \centering
  \includegraphics[scale=0.1]{figure/pacf}
  \caption{\label{fig:pacf} }
\end{figure}

\begin{figure}[ht]
  \centering
  \includegraphics[scale=0.2]{figure/pacf_pairs}
  \caption{\label{fig:pacf_pairs} }
\end{figure}

\begin{figure}[ht]
  \centering
  \includegraphics[scale=0.15]{figure/heatmap-innovations}
  \caption{\label{fig:heatmap-innovations}}
\end{figure}

\begin{figure}[ht]
  \centering
  \includegraphics[scale=0.15]{figure/heatmap-innovations-bin}
  \caption{\label{fig:heatmap-innovations-bin}}
\end{figure}


\begin{figure}[ht]
  \centering
  \includegraphics[scale=0.15]{figure/centroid-innovations-conditional}
  \caption{\label{fig:centroid-innovations-conditional} }
\end{figure}

\section{Hidden Markov Models}

Here we run HMMs on each bacterial time series individually. We arbitrarily set
$K = 4$. An example run on a single bacteria is shown in figure
\ref{fig:hmm-example}. Each row corresponds to a hidden HMM state, and the size
of the point indicates the probability that sample comes from that state. The
model essentially distinguishes between timepoints that have high and low / 0
counts; this is not surprising considering the gaussian emmission structure.

We can run the same analysis across all the bacteria. To view them together, we
need some way of aligning the states with one another. A more sophisticated
approach might try to align the time partitions; we instead naively align them
according to the order of the means in the estimated gaussian emissions. The
results are shown in Figure \ref{fig:hmm-S1} through \ref{fig:hmm-S4}. Each
figure corresponds to a single hidden state. The rows and columns are samples
and bacteria, as before. The shade is the probability that sample / bacteria
combination is in that state. Evidently, state 1 is large outside of the
antibiotics time courses and state 4 is associated with the antibiotics. States
2 and 3 seem like they fit some residual noise.

\begin{figure}[ht]
  \centering
  \includegraphics[scale=0.15]{figure/hmm-example}
  \caption{\label{fig:hmm-example} }
\end{figure}

\begin{figure}[ht]
  \centering
  \includegraphics[scale=0.15]{figure/hmm-S1}
  \caption{\label{fig:hmm-S1} }
\end{figure}

\begin{figure}[ht]
  \centering
  \includegraphics[scale=0.15]{figure/hmm-S2}
  \caption{\label{fig:hmm-S2} }
\end{figure}

\begin{figure}[ht]
  \centering
  \includegraphics[scale=0.15]{figure/hmm-S3}
  \caption{\label{fig:hmm-S3} }
\end{figure}

\begin{figure}[ht]
  \centering
  \includegraphics[scale=0.15]{figure/hmm-S4}
  \caption{\label{fig:hmm-S4} }
\end{figure}

\bibliographystyle{plain}
\bibliography{antibiotic}
\end{document}

