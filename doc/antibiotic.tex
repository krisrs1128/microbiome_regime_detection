\documentclass{article}
\usepackage{natbib}
\usepackage{graphicx}

\title{Hierarchical Clustering + HMM Experiments}
\author{Kris Sankaran}

\begin{document}
\maketitle

These are notes on basic approaches to clustering bacteria. The
three views we consider here are,

\begin{itemize}
\item Hierarchically clustering counts of bacteria, using a few different
  distances, with branches sorted according to abundance. We can then interpret
  the associated heatmaps.
\item Computing the PACF for each bacterial time series. This motivates our
  clustering of the innovations (first-order differences) instead of the raw counts.
\item In addition to clustering bacteria, we can try to cluster the time points.
  For this, we run an HMM on each time series.
\end{itemize}


Our motivation comes from two places,
\begin{itemize}
  \item Applying Latent Dirichlet Allocation on samples seems like a useful
    exercise. Can we carry that way of thinking\footnote{At some point these
      notes should include LDA applied to all the bacteria.} across to the
    bacteria (the transposed matrix)?
  \item A footnote in \citep{ren2015achieving}, presents a probabilistic time
    series clustering model as the unified alternative to first clustering home
    sales across census tracts and then running a kalman filter within each
    cluster. The hope is that some simple exploratory analysis in our setting
    might motivate similar types of unified models.
\end{itemize}

There's really nothing too surprising in what follows. We use the antibiotics
data set as our case study, since we are so familiar with it, though it should
be straightforwards to rerun the code on the cleanout phyloseq object. There are
a few points we've learned in this analysis which might be worth keeping in mind
/ thinking through in follow-up work,

\begin{itemize}
\item Choosing an appropriate distance in hierarchical clustering is
  complicated. We get quite different when using (mixtures of) Euclidean and
  Jaccard distances, which offer somewhat different interpretations. Finding a
  way of simplifying the choice or interpretation would be useful.
\item While taxonomic classification is related to the clustering results,
  limiting attention to just taxonomy would miss out on a lot of interesting
  covariation.
\item The cluster sizes resulting from cutting the hierarchical clustering tree
  are highly imbalanced, which makes interpretation somewhat difficult.
\item Looking at the means in each cluster is misleading, due to the hurdle
  (zero point mass + continuous) nature of the counts.
\item Visually navigating hierarchical clustering results is cumbersome, and it
  might be interesting to develop some interactive alternatives.
\item The PACFs of the bacterial time series suggests that the processes are
  more or less markov, and that clustering innovations might be more informative
  than clustering raw counts.
\item Applying HMMs to the counts is trickier than it might seem at first,
  because (1) label-switching makes it difficult to compare results across
  bacteria without some sort of alignment procedure and (2) running HMM on the
  full collection is not appropriate, because different bacteria have very
  different scales, so different bacteria get marked as coming from different
  regimes even if they have similar shapes.
\end{itemize}

\section{Hierarchical clustering}

\subsection{Euclidean Distance}

\begin{figure}[ht]
  \centering
  \includegraphics[scale=0.15]{figure/heatmap-euclidean}
  \caption{\label{fig:heatmap-euclidean} }
\end{figure}

\begin{figure}[ht]
  \centering
  \includegraphics[scale=0.15]{figure/centroid-euclidean-conditional}
  \caption{\label{fig:centroid-euclidean-conditional} }
\end{figure}

\begin{figure}[ht]
  \centering
  \includegraphics[scale=0.15]{figure/centroid-euclidean-presence}
  \caption{\label{fig:centroid-euclidean-conditional} }
\end{figure}


\subsection{Jaccard Distance}

\begin{figure}[ht]
  \centering
  \includegraphics[scale=0.15]{figure/heatmap-jaccard}
  \caption{\label{fig:heatmap-jaccard} }
\end{figure}

\begin{figure}[ht]
  \centering
  \includegraphics[scale=0.15]{figure/centroid-jaccard-conditional}
  \caption{\label{fig:centroid-jaccard-conditional} }
\end{figure}

\begin{figure}[ht]
  \centering
  \includegraphics[scale=0.15]{figure/centroid-jaccard-presence}
  \caption{\label{fig:centroid-jaccard-conditional} }
\end{figure}

\subsection{Mixture Distance}

\begin{figure}[ht]
  \centering
  \includegraphics[scale=0.15]{figure/heatmap-mix}
  \caption{\label{fig:heatmap-mix} }
\end{figure}

\begin{figure}[ht]
  \centering
  \includegraphics[scale=0.15]{figure/centroid-mix-conditional}
  \caption{\label{fig:centroid-mix-conditional} }
\end{figure}

\begin{figure}[ht]
  \centering
  \includegraphics[scale=0.15]{figure/centroid-mix-presence}
  \caption{\label{fig:centroid-mix-conditional} }
\end{figure}

\bibliographystyle{plain}
\bibliography{antibiotic}
\end{document}

