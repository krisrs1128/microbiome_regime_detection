\documentclass{article}
\usepackage{amsmath, amssymb, amsfonts}
\usepackage{natbib}
\usepackage{graphicx}
\usepackage{hyperref}
\input{preamble.tex}

\title{Inference of Dynamic Regimes in the Microbiome}
\author{Kris Sankaran}
\begin{document}
\maketitle

A typical microbiome data set describes the abundances of bacterial species
across samples. To this point, we have studied latent structure across species
and samples separately. For example, we have developed interactive visualization
techniques to compare subsets of species, and we have applied mixed-membership
models to characterize variation in samples across timepoints. In contrast, our
goal here is to study latent structure across species and samples
simultaneously. This difference is analogous to the change in perspective
obtained by studying a coclustering rather than two separate clusterings, or an
ordination biplot instead of simply the scores or loadings. We will focus on the
case where samples are collected over time, so that this problem can be
understood as one of detecting dynamic regimes, as explained in section
\ref{sec:problem_description}

In this chapter, our contributions are

\begin{itemize}
\item The relation of the regime detection problem to several statistical
  frameworks, and a comparison of the types of interpretation facilitated by
  each.
\item Developing experiments to evaluate the practical utility of these
  different formulations.
\item A catalog of algorithm pseudocode and real implementations, to serve as a
  reference for researchers interested in regime detection.
\item The design of and code for static visualizations that can be used to
  evaluate the results of various methods.
\end{itemize}

compare the behavior of different species, while 

\section{Problem description}
\label{sec:problem_description}

\section{Methods baseline}

\subsection{Hierarchical Clustering}

- Limitations, advantages
- Interpretation of centroids
- Choices: transformation, distances

- Review of hierarchical clustering

- Distances
Using euclidean distance
Using innovations
Using jaccard distance
Combining distances

General question: How to choose between these distances?

Interpretation
- ordinary averages within components
- proportions nonzero

\subsection{Recursive partitioning}

- Variation on hierarchical clustering, to provide temporal structure
- Brief description of recursive partitioning
- Application to the antibiotics data: provide heatmap and interpretation, at
several levels of complexity.

\section{Temporal probabilistic models}

\subsection{Linear dynamical systems}

- Review LDS graphical model
- Derivation of Kalman Filter / Smoother
  + Kalman filtering pseudocode
  + Kalman smoothing pseudocode
  + Link to some actual code

\subsection{Gaussian processes}
- Review GP graphical model
- Basic distinctions from previous models
- Description of GP posterior
  + GP pseudocode
  + Link to some actual code

\section{Temporal mixture models}

\subsection{Hidden Markov Modeling}

- Probabilistic approach that incorporates time structure directly
- Draw the graphical model

\subsubsection{Standard HMMs}

- Write the model and provide an interpretation
- Application to antibiotics: heatmap, interpretation, and transition
probability matrix
- In order to understand extensions, briefly review actual inferential mechanism
- What kinds of limitations have we encountered?

\subsubsection{Sticky HMMs}

- Modified model
- Simulation example
- Antibiotics case study: Transition probabilities and heatmap. Interpretation
of transition probabilities. Contrast with hierarchical clsutering and recursive
partitioning
- Review inferential procedure (block sampling)
  + Link to code
  + Algorithm psueodocode
  + Derivation of some nonobvious steps

\subsubsection{Sticky HDP-HMMs}

- Write the new model
- Simulation example
- Antibiotics case study
- Review inferential procedures (direct assignment and block sampling)
  + Algorithm psueodocode
  + Derivation of some nonobvious steps
  + Link to code

\subsection{Mixture of Experts}

- Switching state space models
   + Model description
   + Simulation example
   + Description of variational inference
- Mixture of Gaussian processes
  + Model description
  + Simulation examples
  + Description of variational inference

\section{Alternative probabilistic models}

\subsection{Changepoint detection}

- Description of the model
- Summary of the empirical bayes idea
- Summary of dynamic programming, and provide row and column updates
  + Maybe provide actual pseudocode?
- Would be nice to include a simulation / application to antibiotics

\subsection{Accounting for zero inflation}

- The dynamic tobit model
  + Brief summary of Glasby Nevison
  + Explanation and simulation using the scan sampler

\end{document}
